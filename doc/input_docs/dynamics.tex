\section{DYNAMICS}
\label{DYNAMICS}
\begin{ipifield}{}%
{Holds all the information for the MD integrator, such as timestep, the thermostats and barostats that control it.}%
{}%
{\ipiitem{mode}%
{The ensemble that will be sampled during the simulation. }%
{default: `nve'; data type: string; options: `nve', `nvt', `npt', `nst', `mts'; }%
}
\begin{ipifield}{timestep}%
{The time step.}%
{dimension: time; default:  1.0 ; data type: float; }%
{\ipiitem{units}%
{The units the input data is given in.}%
{default: `'; data type: string; }%
}
\end{ipifield}
\begin{ipifield}{nmts}%
{Number of iterations for each MTS level (including the outer loop, that should in most cases have just one iteration).}%
{default:  [ ] ; data type: integer; }%
{\ipiitem{shape}%
{The shape of the array.}%
{default:  (0,) ; data type: tuple; }%
}
\end{ipifield}
\begin{ipifield}{\hyperref[THERMOSTATS]{thermostat}}%
{The thermostat for the atoms, keeps the atom velocity distribution at the correct temperature.}%
{}%
{\ipiitem{mode}%
{The style of thermostatting. 'langevin' specifies a white noise langevin equation to be attached to the cartesian representation of the momenta. 'svr' attaches a velocity rescaling thermostat to the cartesian representation of the momenta. Both 'pile\_l' and 'pile\_g' attaches a white noise langevin thermostat to the normal mode representation, with 'pile\_l' attaching a local langevin thermostat to the centroid mode and 'pile\_g' instead attaching a global velocity rescaling thermostat. 'gle' attaches a coloured noise langevin thermostat to the cartesian representation of the momenta, 'nm\_gle' attaches a coloured noise langevin thermostat to the normal mode representation of the momenta and a langevin thermostat to the centroid and 'nm\_gle\_g' attaches a gle thermostat to the normal modes and a svr thermostat to the centroid.  'multiple' is a special thermostat mode, in which one can define multiple thermostats \_inside\_ the thermostat tag.}%
{data type: string; options: `', `langevin', `svr', `pile\_l', `pile\_g', `gle', `nm\_gle', `nm\_gle\_g', `multi'; }%
}
\end{ipifield}
\begin{ipifield}{\hyperref[BAROSTAT]{barostat}}%
{Simulates an external pressure bath.}%
{}%
{\ipiitem{mode}%
{The type of barostat.  Currently, only a 'isotropic' barostat is implemented, that combines
                                   ideas from the Bussi-Zykova-Parrinello barostat for classical MD with ideas from the
                                   Martyna-Hughes-Tuckerman centroid barostat for PIMD; see Ceriotti, More, Manolopoulos, Comp. Phys. Comm. 2013 for
                                   implementation details.}%
{default: `dummy'; data type: string; options: `dummy', `isotropic', `anisotropic'; }%
}
\end{ipifield}
\end{ipifield}
