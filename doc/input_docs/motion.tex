\section{MOTION}
\label{MOTION}
\begin{ipifield}{}%
{Allow chosing the type of calculation to be performed. Holds all the information that is calculation specific, such as geometry optimization parameters, etc.}%
{}%
{\ipiitem{mode}%
{How atoms should be moved at each step in the simulation. 'replay' means that a simulation is restarted from a previous simulation.}%
{data type: string; options: `minimize', `replay', `neb', `dynamics', `dummy'; }%
}
\begin{ipifield}{fixatoms}%
{Indices of the atmoms that should be held fixed.}%
{default:  [ ] ; data type: integer; }%
{\ipiitem{shape}%
{The shape of the array.}%
{default:  (0,) ; data type: tuple; }%
}
\end{ipifield}
\begin{ipifield}{\hyperref[GEOP]{optimizer}}%
{Option for geometry optimization}%
{}%
{\ipiitem{mode}%
{The geometry optimization algorithm to be used}%
{default: `lbfgs'; data type: string; options: `sd', `cg', `bfgs', `lbfgs'; }%
}
\end{ipifield}
\begin{ipifield}{\hyperref[INITFILE]{file}}%
{This describes the location to read a trajectory file from.}%
{default: `'; data type: string; }%
{\ipiitem{mode}%
{The input data format. 'xyz' and 'pdb' stand for xyz and pdb input files respectively. 'chk' stands for initialization from a checkpoint file.}%
{default: `xyz'; data type: string; options: `xyz', `pdb', `chk'; }%
}
\end{ipifield}
\begin{ipifield}{\hyperref[GEOP]{neb\_optimizer}}%
{Option for geometry optimization}%
{}%
{\ipiitem{mode}%
{The geometry optimization algorithm to be used}%
{default: `lbfgs'; data type: string; options: `sd', `cg', `bfgs', `lbfgs'; }%
}
\end{ipifield}
\begin{ipifield}{\hyperref[DYNAMICS]{dynamics}}%
{Option for (path integral) molecular dynamics}%
{}%
{\ipiitem{mode}%
{The ensemble that will be sampled during the simulation. }%
{default: `nve'; data type: string; options: `nve', `nvt', `npt', `nst', `mts'; }%
}
\end{ipifield}
\begin{ipifield}{fixcom}%
{This describes whether the centre of mass of the particles is fixed.}%
{default:  True ; data type: boolean; }%
{}
\end{ipifield}
\end{ipifield}
