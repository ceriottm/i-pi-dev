\section{CHECKPOINT}
\label{CHECKPOINT}
\begin{ipifield}{}%
{This class defines how a checkpoint file should be output. Optionally, between the checkpoint tags, you can specify one integer giving the current step of the simulation. By default this integer will be zero.}%
{data type: integer; }%
{\ipiitem{stride}%
{The number of steps between successive writes.}%
{default:  1 ; data type: integer; }%
\ipiitem{overwrite}%
{This specifies whether or not each consecutive checkpoint file will overwrite the old one.}%
{default:  True ; data type: boolean; }%
\ipiitem{filename}%
{A string to specify the name of the file that is output. The file name is given by 'prefix'.'filename' + format\_specifier. The format specifier may also include a number if multiple similar files are output.}%
{default: `restart'; data type: string; }%
}
\end{ipifield}
